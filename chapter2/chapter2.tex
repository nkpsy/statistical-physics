\newcommand{\diff}[1]{\mathrm{d}#1}
\newcommand{\pd}[3]{\left(\frac{\partial #1}{\partial #2}\right)_{#3}}
\newcommand{\pdoneline}[3]{(\partial #1/\partial #2)_{#3}}

\section{热力学定律,热力学函数}

\textbf{概要:}通常,当我们谈到热力学定律时,我们的大脑中会出现它的三条定律. 事实上,热力学定律有四条. 其中一条是\textbf{第零定律}. 这条定律以第二条基本原理的形式表述(见第1.1节),是一条关于温度的定律. 

在本章中,我们详细阐述了热力学的三条基本定律. 第一定律有关内能及其守恒;第二定律和第三定律有关熵及其变化. 热力学函数的方法和找出它们满足的普遍热力学关系构成了本章的基本内容. 



\subsection{热力学第一定律,功和热量,热容}

我们再次指出,热力学第一定律是关于内能的. 内能是一个宏观系统的状态函数,是守恒的. 众所周知,任何宏观系统会与周围的系统发生相互作用,亦即,与周围的介质. 有以下三条相互作用途径:机械途径($\Delta A \neq 0$),热途径($\Delta Q \neq 0$)和物质途径($\Delta N \neq 0$)(见第1.1节). 

首先,考虑封闭系统($N=\mathrm{const}, \Delta N = 0$)\footnote{\noindent\textrm{开放系统($\Delta N \neq 0$)的热力学将在第2.10节考虑. 译者注:原文为"Sect. 3.1",可能属于印刷错误. }}. 在此情况下,只有两种可能的相互作用类型,也就是机械相互作用和热相互作用. 

\textbf{机械相互作用:功. }考虑一个绝热孤立($\Delta Q = 0$)的封闭系统. 对于这样的系统,只可能存在机械相互作用. 在这种相互作用中,有两种可能的情形:系统消耗内能对外做功,同时体积增大($\Delta V > 0$);系统被施加外力,同时体积减小($\Delta V > 0$). 

当一个系统做功时,其内能降低,因此所做的元功被认为是负值($\diff{A}<0$)\footnote{\noindent\textrm{译者注:原文为"therefore
		the elementary work being performed is regarded as negative",这个负功应该是通常意义上的“外界对气体做的功”. }},类似地,外力对系统所做的元功为正($\diff{A}>0$).

为了计算元功,简单起见,假设一个活塞下的柱形容器中的气体为所考虑的系统(图2.1). 

如果我们使气体的压强为$P$,并且假设活塞的横截面积为$\sigma$,则气体作用于活塞上的力为$P\sigma$,其方向向上. 这个力使活塞上移$\diff{l}$并且做元功$P\sigma\diff{l}=P\diff{V}$. 根据我们的条件,该元功为负. 因此,元功可以用以下形式写出:\footnote{\noindent \textrm{如果在外力的作用下气体的体积减小($\diff{V}<0$),则$\diff{A}>0$. 因此,(2.1)就是功的一般表式. }}
$$	\diff{A}=-P\diff{V},	\eqno{(2.1)}$$
在上述情形中,$\diff{V}>0$. 

可以证明功的表式(2.1)正好适用于任意形式的宏观系统. 为了说明这一点,考虑一个具有任意边界形状(图2.2)的绝热孤立系统. 假设在压强$P$的作用下,系统的体积由$V_1$增大为$V_2$. 为了计算该过程的功,将系统的边界分割为很多小的部分,任意一部分的面积为$\diff{\bm{\sigma}}_i$. 那么,面元$\diff{\bm{\sigma}}_i$受到作用力$P\diff{\bm{\sigma}}_i$. 如果在这个力的作用下面元$\diff{\bm{\sigma}}_i$移动了距离$\diff{\bm{r}}_i$,则所做的元功为$\diff{A}_i=-P\diff{\bm{\sigma}}_i \cdot \diff{\bm{r}}_i=-P\diff{V}_i$,其中$\diff{V}_i=\diff{\bm{\sigma}}_i \cdot \diff{\bm{r}}_i$. 如果我们把系统表面的每一部分作叠加,在气体由体积$V_1$膨胀到$V_2$的过程中,所做的元功可以写为
$$	\diff{A}=\sum_{i} \diff{A}_i=-P\sum_{i}\diff{V}_i=-P\diff{V},	\eqno{(2.2)}	$$
和(2.1)相同. 并且,在外力作用下导致体积由$V_2$减至$V_1$的情形下,仍可得到相同的元功的表式.

由于绝热孤立系统的内能仅在做功时发生改变,因此能量守能定律具有形式
$$	\diff{E}=\diff{A}=-P\diff{V}.	\eqno{(2.3)}$$
如果系统由内能为$E_1$和体积为$V_1$的状态过渡到内能为$E_2$和体积为$V_2$的状态,守恒定律(2.3)可以写为积分形式
$$	E_2-E_1=-\int_{V_1}^{V_2}P(V)\diff{V}.	\eqno{(2.4)}$$
显然,在上述膨胀过程中功和(2.4)中的能量变化在数值上等于$P-V$平面上曲线$P=P(V)$所围成的阴影部分的面积.

\textbf{热相互作用:热量. }考虑一个热源中的具有不可移动边界的系统. 如果热源和系统的温度不同,其间就会发生热量为$\diff{Q}$的热交换. 我们认为从系统传递到热源的热量为负$\diff{Q}<0$,而从热源传递到系统的热量为正$\diff{Q}>0$. 曾经提过,在机械相互作用中,系统的外部参数(体积$V$)发生改变,而在热相互作用中,系统的内部参数(温度$T$)发生改变. 系统和热源的热交换因热传导而发生. 这是一个非常复杂的分子运动过程:系统边界附近的粒子与热源发生相互作用(碰撞)并交换能量,并且,这个碰撞(能量交换)的过程在系统(热源)内部传播. 结果是,一定的热量$\diff{Q}$从系统传递到热源,或者,反之亦然,取决于哪里的温度更高. 由于热量$\diff{Q}$作为交换的结果改变了$\diff{E}$的内能,守恒定律可以表为形式
$$	\diff{E}=\diff{Q}.	\eqno{(2.5)}$$
但是,等式(2.5)仅在热量等价于机械能(功)并且用同一单位度量的情形下才成立. 热和机械能的等价性在1842年由德国生理学家朱利叶斯$\cdot$罗伯特$\cdot$冯$\cdot$迈耶首次确定,并且,在1843年,英国物理学家詹姆斯$\cdot$普雷斯科特$\cdot$焦耳测定了热功当量系数,它是
$$	1\mathrm{\,cal}=4.184\mathrm{\,J}=4.184 \times 10^7 \mathrm{\,erg}.	\eqno{(2.6)}$$
通过对热和功等价性的发现,冯$\cdot$迈耶和焦耳奠定了热力学第一定律的基础. 这条定律的数学表述,由德国物理学家和生理学家赫尔曼$\cdot$路德维希$\cdot$斐迪南$\cdot$冯$\cdot$亥姆霍兹于1847年提出. 因此,三位杰出的科学家,即冯$\cdot$迈耶,焦耳和冯$\cdot$亥姆霍兹,被认为是热力学第一定律的发现者.

假设一个封闭系统由于机械相互作用和热相互作用从初态1过渡到末态2. 通过对这样一种过程的彻底研究,以及大量的关于封闭系统($N=\mathrm{const}$)的实验数据,表明热力学第一定律可以表述为:\textbf{封闭系统由能量为$E_1$的初态1过渡到能量为$E_2$末态2时,能量的改变$(E_2-E_1)$等于由于交换而吸收的热量与所做功的总和;这个总和仅取决于这些状态,而不依赖于系统由初态变化到末态的方式}(图2.4).

由此,可见内能为一状态函数. 因此,热力学第一定律可以写为
$$	\int_{1}^{2} \diff{Q} + \int_{1}^{2} \diff{A}=E_2-E_1.	\eqno{(2.7)}	$$
从第一定律可以得出结论:对于封闭系统,存在一个状态函数(内能$E$),单值地描述系统的状态,并且,其改变量是一个全微分:
$$	\int_{1}^{2} \diff{E}=E_2-E_1.	\eqno{(2.8)}	$$

注意到(2.7)是热力学第一定律的积分形式,即封闭系统的能量守恒定律. 结合(2.7)和(2.8),热力学第一定律的微分形式可以表为形式
$$	\diff{A}+\diff{Q}=\diff{E},	\eqno{(2.9)}	$$
即,\textbf{封闭系统在元过程中的能量改变等于元功与因交换而吸收的热量之和}.

注意一个细微之处. 虽然在(2.9)中$\diff{A}$和$\diff{Q}$均不是全微分,然而其和$\diff{A}+\diff{Q}$,即能量的改变$\diff{E}$却是一个全微分. 这意味着$E$是一个状态函数,而$A$和$Q$不是. 我们可以说一个给定状态的系统的能量$E$是什么,但是我们无法知道哪一部分能量是功,哪一部分是热. 我们只能说,从一个态过渡到另一个态的过程中内能的改变哪一部分导致机械功,哪一部分导致热量.

还存在其他一些热力学第一定律的定义. 如果系统参与了一个循环过程,返回到起始状态,其内能不发生改变(图2.5). 

因此根据(2.7),对于所做的功我们得到
$$	\oint \diff{A}=-\oint \diff{Q},	\eqno{(2.10)}	$$
或者
$$	\oint P(V)\diff{V}=\oint \diff{Q}.	\eqno{(2.11)}	$$
故功在数值上等于循环过程中$P-V$曲线所围成的阴影部分的面积(图2.5),且功的符号取决于循环过程的方向.

从循环过程的热力学第一定律可以得出,系统在吸收一定的热量并且将其转化为等价的功的过程中,会返回到其起始状态.

因此,在循环过程中系统进会在消耗热能的情况下对外做功. 如果$\diff{Q}=0$,从(2.10)可以得出
$$	\oint \diff{A}=0,	\eqno{(2.12)}	$$
即,不花费热能就不可能做功.

曾经有很多年,人们尝试去制造一台不需要消耗热能但能对外做功的机器. 这样的机器被称为“第一类永动机”. 从上面可以看出,热力学第一定律也可以表述为:\textbf{不可能制造第一类永动机}.

\textbf{热容. }在热力学系数中,热容占有特殊的地位. 这里我们仅给出其定义并找出其基于热力学第一定律的普遍表式.

\textbf{热容在数值上等于使物质(系统)温度升高一度所需的热量. 比热容(摩尔热容)在数值上等于使一克(一摩尔)物质温度升高一度所需的热量.}

在实践上,热容通常在两种条件下进行测量:在恒定体积($V=\mathrm{const}$)下,即定容热容$C_V$;在恒定压强($V=\mathrm{const}$)下,即定压热容$C_P$.

利用热力学第一定律的微分形式(2.9),系统所吸收的热量可以写为
$$	\diff{Q}=\diff{E}-\diff{A}.	\eqno{(2.13)}	$$
如果$V,\,T$作为确定系统状态的独立参数,则$E=E(V,T)$. 故能量的改变为
$$	\diff{E}=\pd{E}{V}{T}\diff{V}+\pd{E}{T}{V}\diff{T}.	\eqno{(2.14)}	$$
如果我们将(2.14)和功的表式(2.1)代入(2.13),就得到
$$	\diff{Q}=\pd{E}{T}{V}\diff{T}+\left[\pd{E}{V}{T}+P\right]\diff{V}.	\eqno{(2.15)}	$$
因此,对于定容热容$C_V$我们有
$$	C_V=\pd{Q}{T}{V}=\pd{E}{T}{V}.	\eqno{(2.16)}	$$
在恒定压强$P$下,根据状态方程$V=V(P,T)$,体积$V$仅取决于温度$T$,因此体积的改变为$\diff{V}=(\partial V/\partial T)_{P}\diff{T}$. 那么根据(2.15),在$P=\mathrm{const}$的条件下,热量具有形式
$$	\diff{Q}=\pd{E}{T}{V}\diff{T}+\left[\pd{E}{V}{T}+P\right]\pd{V}{T}{P}\diff{T}.	\eqno{(2.18)}	$$
因此,对于定压热容$C_P=(\partial Q/\partial T)_P$我们有
$$	C_P=C_V+\left[\pd{E}{V}{T}+P\right]\pd{V}{T}{P}.	\eqno{(2.18)}	$$
这里,$\pdoneline{E}{V}{T}\pdoneline{V}{T}{P}$项确定了在恒定压强($P=\mathrm{const}$)下温度改变一度时,系统能量的改变. 而$P\pdoneline{V}{T}{P}$项对应于在$P=\mathrm{const}$,温度改变一度时,花费在用来做功增加体积的热量. 如果系统的能量不取决于体积,即$\pdoneline{E}{V}{T}=0$(理想气体),差值$C_P-C_V=P\pdoneline{V}{T}{P}$由花费在做功上的热量确定.

从(2.16)和(2.18)可以得出,为了计算定容热容,知道系统的热状态方程即能量的依赖关系$E=E(V,T)$已经足够,而为了计算定压热容,还需要知道系统的状态方程$P=P(T,V)$.

我们现在说明热容差值$C_P-C_V$仅由系统的状态方程$P=P(V,T)$确定. 热力学基本关系$\diff{E}=T\diff{S}-P\diff{V}$(见第1.8节)可以表为形式
$$	\pd{E}{V}{T}=T\pd{S}{V}{T}-P.	\eqno(2.19)	$$
在第2.3节中[见(2.62)],我们将证明$\pdoneline{S}{V}{T}=\pdoneline{P}{T}{V}$. 将其代入(2.19),我们得到
$$	\pd{E}{V}{T}+P=T\pd{P}{T}{V}.	\eqno{(2.20)}	$$

由(2.18)和(2.20),$C_P-C_V$最后具有形式
$$	C_P-C_V=T\pd{P}{T}{V}\pd{V}{T}{P}.	\eqno{(2.21)}	$$
从(2.21)可以看出,为了计算差值$C_P-C_V$,知道系统的显式状态方程$P=P(V,T)$已经足够. 尤其是,对于理想气体,利用状态方程$PV=RT$,\footnote{\noindent \textrm{译者注:此处应指1 mol理想气体}}从(2.21)就得到冯$\cdot$迈耶方程:
$$	C_P-C_V=R,	\eqno{(2.22)}	$$
其中$R=N_Ak_0=8.31 \times 10^7 \mathrm{\, erg/(K \cdot mol)}=8.31 \mathrm{\, J/(K \cdot mol)}=1.92 \mathrm{\, cal/(K \cdot mol)}$为普适气体常数,$k_0=1.38 \times 10^{-16} \mathrm{\, erg/K}$为玻尔兹曼常数,而$N_A=6.026 \times 10^{23} \mathrm{\, mol^{-1}}$为阿伏加德罗常数.

可以看出,对于理想气体$C_P>C_V$. 这可以通过以下事实解释,在$P=\mathrm{const}$时,热能中的一部分被消耗用来为气体膨胀做功. 而在$V=\mathrm{const}$的情形下,由于气体没有膨胀,故热能仅被用来增加系统的内能,且因此增加系统的温度. 注意到不等式$C_P>C_V$不仅对于理想气体保持成立,而且对于任意宏观系统也是如此(见第2.4节).



\subsection{热力学第二定律,卡诺循环}

热力学第二定律,即熵增原理的一般表述,在第1.7节已经给出. 熵的统计意义也早在第1.6节就已经给出. 由于一个系统的熵由预先给定的宏观状态对应的微观状态数所确定,或者,换句话说,由预先给定的宏观状态的统计权重的对数确定,故其表征一个系统的状态,且是一个状态函数.

我们介绍克劳修斯于1865年提出的热力学第二定律的定义:\textbf{对于一个不处于热力学平衡状态的孤立系统,其可能的内部过程应该朝着熵增加的方向发展;当系统到达热力学平衡状态时,这些过程停止且熵达到最大值}(图1.8). 

以上是热力学第二定律统计释义的一般形式. 然而,需要注意的是,第二定律和第一定律一样,是一条基于归纳实验数据的实验定律. 在十九世纪初,本着提高热机效率的目标,热和机械功相互转化的过程被深入地加以研究. 实验确立了\textbf{热量$\rightarrow$功(机械能)的转化过程和功$\rightarrow$热量的转化过程是不对称的}. 实验上证实了,所有的机械能(功)可以被转化为等量的热,但是不可能把所有的热完全转化为等量的有效功:
$$	\underrightarrow{\Delta A=\Delta Q}; \quad 	\underrightarrow{\Delta Q>\Delta A}.	\eqno{(2.23)}$$
由此,可以得出,不可能制造一台效率为1的把所有热转化为功的热机,即,不可能制造第二类永动机.

在同一时期,人们开始知道热不能自发地(不做功)由一个冷的物体传递到一个热的物体. 作为归纳以上这些实验结论的结果,出现了两个不同但等价的热力学第二定律的表述:

\textbf{克劳修斯表述}(1850). 

