\section{热力学与统计物理学的基本概念}

\textbf{概要:}本章阐述热力学与统计物理学的基本概念和基本原理。宏观系统包含大量粒子,例如原子、分子、离子、电子、光子和声子等等,本章列举了描述宏观系统状态的不同方法。同时,本章还介绍了微观状态分布函数、一个系统预先给定的宏观状态的统计权重、绝对温度和压强的概念。


\subsection{系统状态的宏观描述:热力学的基本原理}
正如前面所指出的那样,热力学与统计物理学研究具有大量自由度的宏观系统的物理性质。这些系统应该具有足够长的寿命,以便于对它们进行实验观测。包含原子 或分子的通常的气体、光子气体、等离子体、液体和晶体等等都可以作为这些系统的例子。一个很小但仍然是所考虑系统的宏观部分称为\textbf{子系统}。

宏观系统可以与自身也可以与周围的介质发生相互作用,途径包括以下几种:
\begin{enumerate}
	\item 所考虑的系统对其他系统做功,或者反过来,这样的相互作用称为\textbf{机械相互作用}$(\Delta A \neq 0)$。在这种情况下,该系统的体积发生改变。
	\item 仅通过热传递的方式(不做功),使得所考虑系统的能量发生改变,这样的相互作用称为\textbf{热相互作用}$(\Delta Q \neq 0)$。
	\item 系统与系统,或者系统与周围介质之间交换粒子,这样的相互作用称为\textbf{物质相互作用}$(\Delta N \neq 0)$。
\end{enumerate}

自然界中存在各种不同类型的系统,这些不同的类型取决于上述几种途径是开放的或者关闭的。

如果一个系统与周围介质不存在能量和物质交换$(\Delta A = 0,\, \Delta Q = 0,\, \Delta N = 0)$,就称该系统是\textbf{孤立的}。对于这样的系统,所有相互作用的途径都是关闭的。

如果一个系统被一个绝热层所包围,就称该系统为\textbf{绝热孤立系统}$(\Delta Q = 0)$。

如果一个系统不与周围的介质交换粒子$(\Delta N = 0)$,就称这样的系统是\textbf{闭合的}。相反,如果发生了粒子的交换$(\Delta N \neq 0)$,则称这样的系统是\textbf{开放的}。

如果所考虑的系统是一个小的但仍然是一个大系统的宏观部分,那么其内部发生的物理过程就几乎不会影响大系统的热力学状态。在这种情况下,就称这个大系统为\textbf{热源},与其相互作用的系统称为\textbf{热源中的系统}。

对于预先给定外部条件的各系统,其热力学状态可以用有限个可通过实验测定的物理量来描述。这些物理量称为\textbf{热力学参量}。系统的粒子数$N$、其体积$V$、压强$P$、绝对温度$T$、极化强度矢量$\mathscr{P}$、磁化强度矢量$\emph{\textbf{M}}$、电场强度$\bm{\mathcal{E}}$和磁场强度$\bm{\mathit{H}}$均是热力学参量的例子。这些参量描述了系统本身与其所处的外部条件。

